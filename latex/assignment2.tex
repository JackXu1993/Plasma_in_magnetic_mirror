\documentclass[12pt]{article}
\usepackage{amsfonts}
\usepackage{fancyhdr}
\usepackage{comment}
\usepackage[a4paper, top=2.5cm, bottom=2.5cm, left=2.2cm, right=2.2cm]%
{geometry}
\usepackage{times}
\usepackage{amsmath}
\usepackage{changepage}
\usepackage{amssymb}
\usepackage{graphicx}
\usepackage{ listings}
\setcounter{MaxMatrixCols}{30}
\newtheorem{theorem}{Theorem}
\newtheorem{acknowledgement}[theorem]{Acknowledgement}
\newtheorem{algorithm}[theorem]{Algorithm}
\newtheorem{axiom}{Axiom}
\newtheorem{case}[theorem]{Case}
\newtheorem{claim}[theorem]{Claim}
\newtheorem{conclusion}[theorem]{Conclusion}
\newtheorem{condition}[theorem]{Condition}
\newtheorem{conjecture}[theorem]{Conjecture}
\newtheorem{corollary}[theorem]{Corollary}
\newtheorem{criterion}[theorem]{Criterion}
\newtheorem{definition}[theorem]{Definition}
\newtheorem{example}[theorem]{Example}
\newtheorem{exercise}[theorem]{Exercise}
\newtheorem{lemma}[theorem]{Lemma}
\newtheorem{notation}[theorem]{Notation}
\newtheorem{problem}[theorem]{Problem}
\newtheorem{proposition}[theorem]{Proposition}
\newtheorem{remark}[theorem]{Remark}
\newtheorem{solution}[theorem]{Solution}
\newtheorem{summary}[theorem]{Summary}
\newenvironment{proof}[1][Proof]{\textbf{#1.} }{\ \rule{0.5em}{0.5em}}

\newcommand{\Q}{\mathbb{Q}}
\newcommand{\R}{\mathbb{R}}
\newcommand{\C}{\mathbb{C}}
\newcommand{\Z}{\mathbb{Z}}
%%%%%%%%%%%%%%%%%%%%%%%%%%%%%%%%%%%%%%%%%%%%%%%%%%%%%%%%%%%%%%%%%%%%%%%%
\begin{document}

\title{The project of Computational Methods in Physics}
\author{Xu Jue}
\date{\today}
\maketitle
%%%%%%%%%%%%%%%%%%%%%%%%%%%%%%%%%%%%%%%%%%%%%%%%%%%%%%%%%%%%%%%%%%%%%%%%

\section{Principle of the magnetic bottle}
\subsection{Calculate the magnetic field}
\indent

We can get the magnetic field which is generared by a coil(at the origin of the coordinate) by integrating
\begin{equation}
\begin{cases}
        B_{x0}(x,y,z)=\int_0^{2 \pi} \frac{Rzcos \phi}{(x^2+y^2+z^2+R^2-2xRcos \phi-2yRsin \phi)^{3/2}} d\phi\\
        B_{y0}(x,y,z)=\int_0^{2 \pi} \frac{Rzsin \phi}{(x^2+y^2+z^2+R^2-2xRcos \phi-2yRsin \phi)^{3/2}} d\phi\\
        B_{z0}(x,y,z)=\int_0^{2 \pi} \frac{R(R-xcos\phi-ysin\phi)}{(x^2+y^2+z^2+R^2-2xRcos \phi-2yRsin \phi)^{3/2}} d\phi
\end{cases}
\end{equation}


 Then we can get the magnetic field of a magnetic bottle (two coils) by a transformation

 \begin{equation}
 \begin{cases}
        B_{x}(x,y,z)=B_{x0}(x,y,z-d/2)+B_{x0}(x,y,z+d/2)\\
        B_{y}(x,y,z)=B_{y0}(x,y,z-d/2)+B_{y0}(x,y,z+d/2)\\
        B_{z}(x,y,z)=B_{z0}(x,y,z-d/2)+B_{z0}(x,y,z+d/2)
 \end{cases}
 \end{equation}
 d in these equations means the distance between two coils.


\subsection{Calculate the movement of the electron}
According to the Newton equation and Lorentz force equation
\begin{equation}
\begin{cases}
    F=ma\\
    F_{Lorentz}=qv \times x
\end{cases}
\end{equation}
Then we can get the equation,
\begin{equation}
    qv \times x=ma
\end{equation}

Then we need to solve the second order Ordinary Differential Equations, because we study this problem in 3D.
\begin{equation}
    \begin{cases}
        x''=-\frac{m}{q}(y'Bz(x,y,z)-z'By(x,y,z))\\
        y''=-\frac{m}{q}(z'Bx(x,y,z)-x'Bz(x,y,z))\\
        z''=-\frac{m}{q}(x'By(x,y,z)-y'Bx(x,y,z))
    \end{cases}
\end{equation}

We transfer the them into first order Ordinary Differential Equations,

\begin{equation}
    \begin{cases}
        vx'=-sc(vy*Bz(x,y,z)-vz*By(x,y,z))\\
        vy'=-sc(vz*Bx(x,y,z)-vx*Bz(x,y,z))\\
        vz'=-sc(vx*By(x,y,z)-vy*Bx(x,y,z))\\
        x'=vx\\
        y'=vy\\
        z'=vz
    \end{cases}
\end{equation}

Its intial values are

\begin{equation}
    \begin{cases}
        vx(0)=0\\
        vy(0)=0\\
        vz(0)=0.15e6\\
        x(0)=0\\
        y(0)=0.78R\\
        z(0)=-0.75d
    \end{cases}
\end{equation}
%%%%%%%%%%%%%%%%%%%%%%%%%%%%%%%%%%%%%%%%%%%%%%%%%%%%%%%%%%%%%%%%%%%%%%%%
\subsection{Solution}
%%%%%%%%%%%%%%%%%%%%%%%%%%%%%%%%%%%%%%%%%%%%%%%%%%%%%%%%%%%%%%%%%%%%%%%%
\subsubsection{Equation to be solved}

\indent


%%%%%%%%%%%%%%%%%%%%%%%%%%%%%%%%%%%%%%%%%%%%%%%%%%%%%%%%%%%%%%%%%%%%%%%%
\subsubsection{Numerical method used}

%%%%%%%%%%%%%%%%%%%%%%%%%%%%%%%%%%%%%%%%%%%%%%%%%%%%%%%%%%%%%%%%%%%%%%%%
\subsubsection{Results}


%\begin{figure}[!h]
%\centering
%\includegraphics[width=4in]{2.pdf}
%\caption{square finite well and wave functions}
%\label{gasbulbdata}
%\end{figure}

%%%%%%%%%%%%%%%%%%%%%%%%%%%%%%%%%%%%%%%%%%%%%%%%%%%%%%%%%%%%%%%%%%%%%%%%
\subsubsection{Discussions}

%%%%%%%%%%%%%%%%%%%%%%%%%%%%%%%%%%%%%%%%%%%%%%%%%%%%%%%%%%%%%%%%%%%%%%%%


\end{document}
\newcommand{\Z}{\mathbb{Z}}
